\documentclass[a4paper, 12pt]{article}

\title{Drinking Water Becoming a service instead of a right}
\author{Ayush Dutta}

\begin{document}
\maketitle

\section{Introduction}
Drinking water in cities has now become something that you need to buy in the same way as you buy your groceries. But the biggest difference is that for groceries you have a lot of options to buy from. There are big marts like
\textit{Big bazaar and D-Marts}, 
then you have your local shopkeepers as well. An interesting thing to note here is that now a lot of online platforms are available for delivering your groceries or fruits at your door steps. Some of the big giants like
\textit{Flipkart and Amazon} 
have also started delivering groceries. But none of the companies are there for providing fresh drinking water. So in this paper we will be trying to solve this problem and come out with a practical and modern solution because in future we will be having a lot of problems regarding the drinking water if we are not prepared for it right now.

\section{Case Study: Indore}
In urban cities like Indore, there are widely three types of drinking water supply system working majorly.
\begin{enumerate}
    \item Through pipelines
    \item Through Nagar Palika tankers
    \item By local vendors
\end{enumerate}

\subsection[short]{Let us take the example of Indore city and discuss further}
In indore city the municipal charges \textbf{Rs. 50.68} per 1000 litres. which for a month becomes on an average of rs. 152 for regular usage but in case of extra usage 
that actually happens during summer seasons the cost may go high as \textit{rs. 200} per month. on the other hand the water that comes through the pipelines as well as 
the municipal tankers are not actually fit for drinking. recent tests in the water supply of the city has proved that the water that is provided from \textit{Narmada} is 
fine for drinking but the ones provided from the boring is not advised to use directly for drinking purpose. Rather use of R.O is mendatory for that. the other key thing 
to note here is that the water from \textit{Narmada} is not in wide use. the use is very limited and currently government is working for making it available for the rest
of the city as well. But it will take a little more time than expected. 

\subsection[short]{Option currently available}
A lot of local water suppliers have now started throughout the city to provide the drinking water to the households. These water suppliers provide water in water cans 
of different volumes like 3L can, 2L can etc. But the problem with this system is its very discontinous or we can say that the distribution of these local vendors are
unevenly distributed that makes it difficult for people living in other areas of the city. How this system works is another point of pain for the people. The local vendors
don't follow a systematic plan. what they basically do is they give their contact number to the people who will buy the subscription and then the people will call them and
ask for the volume of water they require. now the thing is it becomes really difficult to arrange the phone numbers of these vendors. 
\subparagraph*{So on summarising the overall problems}

\begin{enumerate}
    \item Its very difficult for the person who is new in the city to find water suppliers.
    \item People have to search for the phone numbers of these suppliers as they dont have their website or app.
    \item For people, there are very few options as they dont know about other water suppliers since there is no information available regarding this.
\end{enumerate}

\section*{Discussing the solution}
To solve this issue we have figured out a couple of things that will try to solve this issue. we have been analysing the business
models of \textbf{Zomato} and \textbf{Swiggy} and we have found that our solution can be based on a similar business model as that 
of zomato and swiggy. Basically we are trying to make a platform or in other words \textit{a medium} between the customers and the 
local vendors.

\subsection[short]{Explaining the platform}
we are currently thinking of making a website that will be having two options provided to the customers and the suppliers respectively.
these two options are as follows.
\begin{enumerate}
    \item Register as customer.
    \item Register as supplier.
\end{enumerate}
so the customers need to register from the first option where they will be giving their location and we will help them in finding
out the number of water suppliers present nearby. that information will include their costing, how much they charge per litres of water,
what are their other schemes and what are their ratings. there they will be having an option to -
\begin{enumerate}
    \item Ask for an immediate order (if they want to get water delivered for a day or two).
    \item Get a monthly subscription (if they want to get the supply for a month).
\end{enumerate}
in this the customers will fill their details about how much litres of water they want and which supplier suits their budget. Now talking about the other option provided
to the supplier. suppliers will register their details on our site so that their information is available to a lot more people where earlier only the people nearby their
location used to know. it will not only market their business but also will help them to go \textit{digital}. While registering they will be entering all their details
including their
\begin{enumerate}
    \item Bank details (because the customers will be paying them through us)
    \item Contact details (in case of any emergency)
    \item If they have any delivering vehicles or not.
    \item Capacity of production, etc.
\end{enumerate}

\subsection*{Is this idea really scalable?}
As of now, at the initial state of our venture we are not focusing on producing the service by ourself. The cloud kitchen's like idea is what we are planning right now
for the future of this business. The way cloud kitchen works is it is a kind of a restaurant from where we can order food which would be later delivered to us by the 
respective company that runs the cloud kitchen. An interesting thing to note here is that you cannot physically go and eat there. So basically its kind of an own manufacturing
unit of swiggy/zomato. instead of taking orders to other restaurants they are doing it by themselves. The same principle can be applied to this business as well. Once
we reach at a state where we have enough customer and enough finance we can open our own water purifying center, where we can provide better assistance and service to the
customers. Not only that, we will be eliminating any other middle party that would currently be present in the initial days. This can furthur be converted to much more 
wider business. I have read few reports of the water crisis happening in various cities. People who cannot afford packaged water have to go miles to fillup their water containers
which is provided by the Local Govt. Authority. 






\end{document}
